\documentclass[a4paper]{article}
\usepackage{geometry}
\usepackage{graphicx}
\usepackage{natbib}
\usepackage{amsmath}
\usepackage{amssymb}
\usepackage{amsthm}
\usepackage{paralist}
\usepackage{epstopdf}
\usepackage{tabularx}
\usepackage{longtable}
\usepackage{multirow}
\usepackage{multicol}
\usepackage[hidelinks]{hyperref}
\usepackage{fancyvrb}
\usepackage{algorithm}
\usepackage{algorithmic}
\usepackage{float}
\usepackage{paralist}
\usepackage[svgname]{xcolor}
\usepackage{enumerate}
\usepackage{array}
\usepackage{times}
\usepackage{url}
\usepackage{fancyhdr}
\usepackage{comment}
\usepackage{environ}
\usepackage{times}
\usepackage{textcomp}
\usepackage{caption}
\usepackage{multirow}
\usepackage{bbm}

% \usepackage{kky}

\newcounter{thm}
\ifx\fact\undefined
\newtheorem{fact}[thm]{Fact}
\fi

\newcommand{\pen}{{\rm pen}}
\newcommand{\diag}{{\rm diag}}
\newcommand{\diam}{{\bf{\rm diam}}}
\newcommand{\spann}{{\bf{\rm span}}}
\newcommand{\nulll}{{\bf{\rm null}}}
% Distributions
\newcommand{\Bern}{{\bf{\rm Bern}}\,} % support of a function
\newcommand{\Categ}{{\bf{\rm Categ}}\,} % support of a function
\newcommand{\Mult}{{\bf{\rm Mult}}\,} % support of a function
\newcommand{\Dir}{{\bf{\rm Dir}}\,} % support of a function
\newcommand{\horizontalline}{\noindent\rule[0.5ex]{\linewidth}{1pt}}
\newcommand{\HRule}{\rule{\linewidth}{0.5mm}} 
\newcommand{\Hrule}{\rule{\linewidth}{0.3mm}}
\newcommand{\HRuleN}{\HRule\\} 
\newcommand{\HruleN}{\Hrule\\}
\newcommand{\superscript}[1]{{\scriptsize \ensuremath{^{\textrm{#1}}}}}
\newcommand{\supindex}[2]{#1^{(#2)}}
\newcommand{\xii}[1]{\supindex{x}{#1}}
\newcommand{\yii}[1]{\supindex{y}{#1}}
\newcommand{\zii}[1]{\supindex{z}{#1}}
\newcommand{\Xii}[1]{\supindex{X}{#1}}
\newcommand{\Yii}[1]{\supindex{Y}{#1}}
\newcommand{\Zii}[1]{\supindex{Z}{#1}}
\newcommand{\NN}{\mathbb{N}} % Natural numbers
\newcommand{\Ncal}{\mathcal{N}}
\newcommand{\Dcal}{\mathcal{D}}
\newcommand{\Lcal}{\mathcal{L}}
\newcommand{\Xcal}{\mathcal{X}}
\newcommand{\Pcal}{\mathcal{P}}
\newcommand{\Jcal}{\mathcal{J}}
\newcommand{\Rcal}{\mathcal{R}}
\newcommand{\indfone}{\mathbbm{1}}
\newcommand{\gb}{\mathbf{g}}
\newcommand{\Hb}{\mathbf{H}}
\newcommand{\Db}{\mathbf{D}}
\newcommand*{\zero}{{\bf 0}}
\newcommand*{\one}{{\bf 1}}

% Stuff mostly appearing in Statistics
\newcommand{\Xbar}{\bar{X}}
\newcommand{\Ybar}{\bar{Y}}
\newcommand{\Zbar}{\bar{Z}}
\newcommand{\Xb}{\mathbf{X}}


%%%%  brackets
\newcommand{\inner}[2]{\left\langle #1,#2 \right\rangle}
\newcommand{\rbr}[1]{\left(#1\right)}
\newcommand{\sbr}[1]{\left[#1\right]}
\newcommand{\cbr}[1]{\left\{#1\right\}}
\newcommand{\nbr}[1]{\left\|#1\right\|}
\newcommand{\abr}[1]{\left|#1\right|}

% derivatives and partial fractions
\newcommand{\differentiate}[2]{ \frac{ \ud #2}{\ud #1} }
\newcommand{\differentiateat}[3]{ \frac{ \ud #2}{\ud #1}  \Big|_{#1=#3} }
\newcommand{\partialfrac}[2]{ \frac{ \partial #2}{\partial #1} }
\newcommand{\partialfracat}[3]{ \frac{ \partial #2}{\partial #1} \Big|_{#1=#3} }
\newcommand{\partialfracorder}[3]{ \frac{ \partial^{#3} #2}{\partial^{#3} #1} }
\newcommand{\partialfracatorder}[4]{ \frac{ \partial^{#3} #2}{\partial^{#3} #1} \Big|_{#1=#4} }

\urlstyle{rm}

\setlength\parindent{0pt} % Removes all indentation from paragraphs
\theoremstyle{definition}
\newtheorem{definition}{Definition}[]
\newtheorem{conjecture}{Conjecture}[]
\newtheorem{example}{Example}[]
\newtheorem{theorem}{Theorem}[]
\newtheorem{lemma}{Lemma}
\newtheorem{proposition}{Proposition}
\newtheorem{corollary}{Corollary}


\floatname{algorithm}{Procedure}
\renewcommand{\algorithmicrequire}{\textbf{Input:}}
\renewcommand{\algorithmicensure}{\textbf{Output:}}
\newcommand{\abs}[1]{\lvert#1\rvert}
\newcommand{\norm}[1]{\lVert#1\rVert}
\newcommand{\RR}{\mathbb{R}}
\newcommand{\EE}{\mathbb{E}}
\newcommand{\PP}{\mathbb{P}}
\newcommand{\CC}{\mathbb{C}}
\newcommand{\Nat}{\mathbb{N}}
\newcommand{\br}[1]{\{#1\}}
\DeclareMathOperator*{\argmin}{arg\,min}
\DeclareMathOperator*{\argmax}{arg\,max}
\renewcommand{\qedsymbol}{$\blacksquare$}

\definecolor{dkgreen}{rgb}{0,0.6,0}
\definecolor{gray}{rgb}{0.5,0.5,0.5}
\definecolor{mauve}{rgb}{0.58,0,0.82}

\newcommand{\Var}{\mathrm{Var}}
\newcommand{\Cov}{\mathrm{Cov}}

\newcommand{\vc}[1]{\boldsymbol{#1}}
\newcommand{\xv}{\vc{x}}
\newcommand{\Sigmav}{\vc{\Sigma}}
\newcommand{\alphav}{\vc{\alpha}}
\newcommand{\muv}{\vc{\mu}}

\newcommand{\red}[1]{\textcolor{red}{#1}}

\def\x{\mathbf x}
\def\y{\mathbf y}
\def\w{\mathbf w}
\def\v{\mathbf v}
\def\E{\mathbb E}
\def\V{\mathbb V}

\hypersetup{
%    colorlinks,
    linkcolor={red!50!black},
    citecolor={blue!50!black},
    urlcolor={blue!80!black}
}

\geometry{
  top=1in,            % <-- you want to adjust this
  inner=1in,
  outer=1in,
  bottom=1in,
  headheight=3em,       % <-- and this
  headsep=2em,          % <-- and this
  footskip=3em,
}


\pagestyle{fancyplain}
\lhead{\fancyplain{}{Homework 5}}
\rhead{\fancyplain{}{CS 760 Machine Learning}}
\cfoot{\thepage}

\title{\textsc{Homework 5}} % Title

%%% NOTE:  Replace 'NAME HERE' etc., and delete any "\red{}" wrappers (so it won't show up as red)

\author{
\red{$>>$Huzaifa Mustafa Unjhawala$<<$} \\
\red{$>>$GITHUB LINK - https://github.com/Huzaifg/760.git$<<$}\\
}

\date{}

\begin{document}

\maketitle 


\textbf{Instructions:}
Use this latex file as a template to develop your homework. Submit your homework on time as a single pdf file. Please wrap your code and upload to a public GitHub repo, then attach the link below the instructions so that we can access it. Answers to the questions that are not within the pdf are not accepted. This includes external links or answers attached to the code implementation. Late submissions may not be accepted. You can choose any programming language (i.e. python, R, or MATLAB). Please check Piazza for updates about the homework. It is ok to share the experiments results and compare them with each other.

\vspace{0.1in}


\section{Clustering}

\subsection{K-means Clustering (14 points)}

\begin{enumerate}

\item \textbf{(6 Points)}
Given $n$ observations $X_1^n = \{X_1, \dots, X_n\}$, $X_i \in \Xcal$, the K-means objective
is to find $k$
($<n$) centres $\mu_1^k = \{\mu_1, \dots, \mu_k\}$, and a rule $f:\Xcal \rightarrow
\{1,\dots, K\}$ so as to minimize the objective

\begin{equation}
J(\mu_1^K, f; X_1^n) = \sum_{i=1}^n \sum_{k=1}^K \indfone(f(X_i) = k) \|X_i - \mu_k\|^2
\label{eqn:kmeans}
\end{equation}

Let $\Jcal_K(X_1^n) = \min_{\mu_1^K, f} J(\mu_1^K, f; X_1^n)$. Prove that
$\Jcal_{K}(X_1^n)$ is a non-increasing function of $K$.\\

\textbf{Solution: }\\
Let $K_1 < K_2$ and $\mu_1^{K_1}, f_1$ and $\mu_1^{K_2}, f_2$ be the optimal solutions for $K_1$ and $K_2$ respectively.\\
We can write the objective function as:
\begin{align*}
    J(\mu_1^{K_1}, f_1; X_1^n) &= \sum_{i=1}^n \sum_{k=1}^{K_1} \indfone(f_1(X_i) = k) \|X_i - \mu_k\|^2\\
    &= \sum_{i=1}^n \sum_{k=1}^{K_1} \indfone(f_1(X_i) = k) \|X_i - \mu_k\|^2 + \sum_{i=1}^n \sum_{k=K_1+1}^{K_2} \indfone(f_1(X_i) = k) \|X_i - \mu_k\|^2\\
    &\geq \sum_{i=1}^n \sum_{k=1}^{K_2} \indfone(f_2(X_i) = k) \|X_i - \mu_k\|^2\\
    &= J(\mu_1^{K_2}, f_2; X_1^n)
\end{align*}
Therefore, $\Jcal_{K}(X_1^n)$ is a non-increasing function of $K$.

\item \textbf{(8 Points)}
Consider the K-means (Lloyd's) clustering algorithm we studied in class. We
terminate the algorithm when there are no changes to the objective.
Show that the algorithm terminates in a finite number of steps. \\

\end{enumerate}

\textbf{Solution: }\\
Due to the nature of the K-Means algorithm, we always assign a cluster center and update the cluster centers that minimize the K-means objective function. Thus, Lloyd's algorithm will never move to a new configuration of cluster centers and assignments that have a higher objective function value. Since there are only a finite number of assignment configurations, and with each new assignment configuration our objective value decreases, in a finite number of steps, we will either run out of assignment configurations or have our objective value stagnate at a minima. Thus, K-means will terminate in a finite number of steps.




\subsection{Experiment (20 Points)}

In this question, we will evaluate
K-means clustering and GMM on a simple 2 dimensional problem.
First, create a two-dimensional synthetic dataset of 300 points by sampling 100 points each from the
three Gaussian distributions shown below:
\[
P_a = \Ncal\left(
\begin{bmatrix}
-1 \\ -1
\end{bmatrix},
\;
\sigma\begin{bmatrix}
2, &0.5 \\ 0.5, &1
\end{bmatrix}
\right),
\quad
P_b = \Ncal\left(
\begin{bmatrix}
1 \\ -1
\end{bmatrix},
\;
 \sigma\begin{bmatrix}
1, &-0.5 \\ -0.5, &2
\end{bmatrix}
\right),
\quad
P_c = \Ncal\left(
\begin{bmatrix}
0 \\ 1
\end{bmatrix},
\;
 \sigma\begin{bmatrix}
1 &0 \\ 0, &2
\end{bmatrix}
\right)
\]
Here, $\sigma$ is a parameter we will change to produce different datasets.\\

First implement K-means clustering and the expectation maximization algorithm for GMMs.
Execute both methods on five synthetic datasets,
generated as shown above with $\sigma \in \{0.5, 1, 2, 4, 8\}$. Finally, evaluate both methods on \emph{(i)} the clustering objective~\eqref{eqn:kmeans} and \emph{(ii)}  the clustering accuracy. For each of the two criteria, plot the value achieved by each method against $\sigma$.\\


Guidelines:
\begin{itemize} 
\item Both algorithms are only guaranteed to find only a local optimum so we recommend trying multiple
restarts and picking the one with the lowest objective value (This is~\eqref{eqn:kmeans} for K-means and the negative log likelihood for GMMs).
You may also experiment with a smart initialization
strategy (such as kmeans++).

\item
To plot the clustering accuracy,  you may treat the `label' of points generated from distribution
$P_u$ as $u$, where $u\in \{a, b, c\}$.
Assume that the cluster id $i$ returned by a method is $i\in \{1, 2, 3\}$.
Since clustering is an unsupervised learning problem, you should obtain the best possible mapping
from $\{1, 2, 3\}$ to $\{a, b, c\}$ to compute the clustering objective.
One way to do this is to compare the clustering centers returned by the method (centroids for
K-means, means for GMMs) and map them to the distribution with the closest mean.

\end{itemize}

Points break down: 7 points each for implementation of each method, 6 points for reporting of
evaluation metrics.

\textbf{Solution: }\\
\begin{figure}[H]
    \centering
    \includegraphics[width=0.5\linewidth]{../images/1_2_clustering_obj.png}
    \caption{Clustering Objective}
    \label{fig:obj}
\end{figure}


\begin{figure}[H]
    \centering
    \includegraphics[width=0.5\linewidth]{../images/1_2_clustering_acc.png}
    \caption{Clustering Accuracy}
    \label{fig:acc}
\end{figure}

\section{Linear Dimensionality Reduction}

\subsection{Principal Components Analysis  (10 points)}
\label{sec:pca}

Principal Components Analysis (PCA) is a popular method for linear dimensionality reduction. PCA attempts to find a lower dimensional subspace such that when you project the data onto the subspace as much of the information is preserved. Say we have data $X = [x_1^\top; \dots; x_n^\top] \in \RR^{n\times D}$ where  $x_i \in \RR^D$. We wish to find a $d$ ($ < D$) dimensional subspace $A = [a_1, \dots, a_d] \in \RR^{D\times d}$, such that $ a_i \in \RR^D$ and $A^\top A = I_d$, so as to maximize $\frac{1}{n} \sum_{i=1}^n \|A^\top x_i\|^2$.
\begin{enumerate}

\item  \textbf{(4 Points)}
Suppose we wish to find the first direction $a_1$ (such that $a_1^\top a_1 = 1$) to maximize $\frac{1}{n} \sum_i (a_1^\top x_i)^2$.
Show that $a_1$ is the first right singular vector of $X$.

\item  \textbf{(6 Points)}
Given $a_1, \dots, a_k$, let $A_k = [a_1, \dots, a_k]$ and 
$\tilde{x}_i = x_i - A_kA_k^\top x_i$. We wish to find $a_{k+1}$, to maximize
$\frac{1}{n} \sum_i (a_{k+1}^\top \tilde{x}_i)^2$. Show that $a_{k+1}$ is the
$(k+1)^{th}$ right singular vector of $X$.


\end{enumerate}

\textbf{Solution: }\\

\begin{enumerate}
    \item We can write the objective function as:
    \begin{align*}
        \frac{1}{n} \sum_i (a_1^\top x_i)^2 &= \frac{1}{n} \sum_i (a_1^\top x_i)(x_i^\top a_1)\\
        &= \frac{1}{n} \sum_i a_1^\top x_i x_i^\top a_1\\
        &= a_1^\top \left(\frac{1}{n} \sum_i x_i x_i^\top \right) a_1\\
        &= a_1^\top \left(\frac{1}{n} X^\top X \right) a_1
    \end{align*}
    We would like to maximize this subject to the constraint that $a_1^Ta_1 = 1$. Thus we can write the Lagrangian as:
    \begin{align*}
        \mathcal{L}(a_1, \lambda) &= a_1^\top \left(\frac{1}{n} X^\top X \right) a_1 - \lambda(a_1^Ta_1 - 1)\\
        \frac{\partial \mathcal{L}}{\partial a_1} &= \frac{2}{n} X^\top X a_1 - 2\lambda a_1 = 0\\
        \implies X^\top X a_1 &= \lambda a_1
    \end{align*}
    Thus, $a_1$ is the first right singular vector of $X$ as the last line is the definition of the right singular vector.

    \item Note that $\tilde{X} = X - A_kA_k^TX$. Then, using a similar approach, we can write the objective function as:
    \begin{align*}
        \frac{1}{n} \sum_i (a_{k+1}^\top \tilde{x}_i)^2 &= \frac{1}{n} \sum_i (a_{k+1}^\top \tilde{x}_i)(\tilde{x}_i^\top a_{k+1})\\
        &= \frac{1}{n} \sum_i a_{k+1}^\top \tilde{x}_i \tilde{x}_i^\top a_{k+1}\\
        &= a_{k+1}^\top \left(\frac{1}{n} \sum_i \tilde{x}_i \tilde{x}_i^\top \right) a_{k+1}\\
        &= a_{k+1}^\top \left(\frac{1}{n} \tilde{X}^\top \tilde{X} \right) a_{k+1} \\
    \end{align*}
    We would like to maximize this subject to the constraint that $a_{k+1}^Ta_{k+1} = 1$. Thus we can write the Lagrangian as:
    \begin{align*}
        \mathcal{L}(a_{k+1}, \lambda) &= a_{k+1}^\top \left(\frac{1}{n} \tilde{X}^\top \tilde{X} \right) a_{k+1} - \lambda(a_{k+1}^Ta_{k+1} - 1)\\
        \frac{\partial \mathcal{L}}{\partial a_{k+1}} &= \frac{2}{n} \tilde{X}^\top \tilde{X} a_{k+1} - 2\lambda a_{k+1} = 0\\
        \implies \tilde{X}^\top \tilde{X} a_{k+1} &= \lambda a_{k+1}
    \end{align*}
    Now note that $\tilde{X} = (\mathbb{I} - A_kA_k^T)X$. Let $(\mathbb{I} - A_kA_k^T) = E$, thus $\tilde{X} = EX$. Using this in the above equations:
    \begin{align*}
        \tilde{X}^\top \tilde{X} a_{k+1} &= \lambda a_{k+1}\\
        \implies X^\top E^\top E X a_{k+1} &= \lambda a_{k+1}\\
        \implies X^\top X a_{k+1} &= \lambda a_{k+1}
    \end{align*}
    Thus, $a_{k+1}$ is the $(k+1)^{th}$ right singular vector of $X$ as the last line is the definition of the right singular vector.

\end{enumerate}


\subsection{Dimensionality reduction via optimization (22 points)}

We will now motivate the dimensionality reduction problem from a slightly different
perspective. The resulting algorithm has many similarities to PCA.
We will refer to method as DRO.

As before, you are given data $\{x_i\}_{i=1}^n$, where $x_i \in \RR^D$. Let $X=[x_1^\top; \dots
x_n^\top] \in \RR^{n\times D}$. We suspect that the data
actually lies approximately in  a $d$ dimensional affine subspace.
Here $d<D$ and $d<n$.
Our goal, as in PCA, is to use this dataset to find a $d$ dimensional representation $z$ for each $x\in\RR^D$.
(We will assume that the span of the data has dimension larger than
$d$, but our method should work whether $n>D$ or $n<D$.)


Let $z_i\in \RR^d$ be the lower dimensional representation for $x_i$ and
let $Z = [z_1^\top; \dots; z_n^\top] \in \RR^{n\times d}$.
We wish to find parameters $A \in \RR^{D\times d}$, $b\in\RR^D$ and the lower
dimensional representation $Z\in \RR^{n\times d}$ so as to minimize 
\begin{equation}
J(A,b,Z) = \frac{1}{n} \sum_{i=1}^n \|x_i - Az_i - b\|^2 = \| X - ZA^\top - \one b^\top\|_F^2.
\label{eqn:dimobj}
\end{equation}
Here, $\|A\|^2_F = \sum_{i,j} A_{ij}^2$ is the Frobenius norm of a matrix.


\begin{enumerate}
\item \textbf{(3 Points)}
Let $M\in\RR^{d\times d}$ be an arbitrary invertible matrix and $p\in\RR^{d}$ be an arbitrary vector.
Denote, $A_2 = A_1M^{-1}$, $b_2 = b_1- A_1M^{-1}p$ and $Z_2 = Z_1 M^\top +
\one p^\top$.
Show that both
$(A_1, b_1, Z_1)$ and $(A_2, b_2, Z_2)$ achieve the same objective value $J$~\eqref{eqn:dimobj}.
\end{enumerate}

Therefore, in order to make the problem determined, we need to impose some
constraint on $Z$. We will assume that the $z_i$'s have zero mean and identity covariance.
That is,
\begin{align*}
\Zbar = \frac{1}{n} \sum_{i=1}^n z_i =\frac{1}{n} Z^\top {\bf 1}_n = 0, \hspace{0.3in} 
S = \frac{1}{n} \sum_{i=1}^n z_i z_i^\top 
= \frac{1}{n} Z^\top Z
= I_d
\end{align*}
Here, ${\bf 1}_d = [1, 1 \dots, 1]^\top \in\RR^d$ and $I_d$  is the $d\times d$ identity matrix.

\begin{enumerate}
\setcounter{enumi}{1}
\item \textbf{(16 Points)}
Outline a procedure to solve the above problem. Specify how you
would obtain $A, Z, b$ which minimize the objective and satisfy the constraints.

\textbf{Hint: }The rank $k$ approximation of a matrix in Frobenius norm is obtained by
taking its SVD and then zeroing out all but the first $k$ singular values.

\item \textbf{(3 Points)}
You are given a point $x_*$ in the original $D$ dimensional space.
State the rule to obtain the $d$ dimensional
representation $z_*$ for this new point.
(If $x_*$ is some original point $x_i$ from the $D$--dimensional space, it shoud be the
$d$--dimensional representation $z_i$.)


\end{enumerate}

\textbf{Solution: }
\begin{enumerate}
    \item We can write the objective function as:
    \begin{align*}
        J(A_2, b_2, Z_2) &= \| X - Z_2A_2^\top - \one b_2^\top\|_F^2\\
        &= \| X - (Z_1 M^\top + \one p^\top)A_1^\top - \one (b_1- A_1M^{-1}p)^\top\|_F^2\\
        &= \| X - Z_1M^\top A_1^\top - \one b_1^\top + \one A_1M^{-1}p^\top\|_F^2\\
        &= \| X - Z_1A_1^\top - \one b_1^\top\|_F^2\\
        &= J(A_1, b_1, Z_1)
    \end{align*}
    Thus, both $(A_1, b_1, Z_1)$ and $(A_2, b_2, Z_2)$ achieve the same objective value $J$~\eqref{eqn:dimobj}.

    \item There are multiple ways to solve this problem. One way would be to optimize the objective function with respect to $A, b, Z$ and then project $Z$ onto the subspace spanned by the first $d$ singular vectors of $Z$. We could add the constraints stated above to this optimization problem and solve using standard optimization algorithms like L-BFGS.\\
    The other way would be to optimize the objection function for $b$. This means taking the gradient of the objective function with respect to $b$ and setting it to zero. This gives us:
    \begin{align*}
        \frac{\partial}{\partial b} \left( \frac{1}{n} \sum_{i=1}^n \|x_i - Az_i - b\|^2 \right) &= 0 \\
        \implies \frac{1}{n} \sum_{i=1}^n -2(x_i - Az_i - b) &= 0\\
    \end{align*}
    Now, we know that $\frac{1}{n} \sum_{i=1}^n z_i = 0$, because $z_i$ is zero mean. Thus, we can write the above equation as:
    \begin{align*}
        \frac{1}{n} \sum_{i=1}^n -2(x_i - Az_i - b) &= 0\\
        \implies \frac{1}{n} \sum_{i=1}^n -2(x_i - Az_i) - 2b &= 0\\
        \implies \frac{1}{n} \sum_{i=1}^n -2(x_i - Az_i) &= 2b\\
        \implies b &= \frac{1}{n} \sum_{i=1}^n (x_i - Az_i) \\
        \implies b &= \frac{1}{n} \sum_{i=1}^n x_i
    \end{align*}
    Now, that we know $b$, we can define $Y = X - b$ to get mean centered data. Then, to minimize our function $\frac{1}{n} \| Y - ZA^\top\|_F^2$. 
    Taking the derivative of this function with respect to $A$, we get:
    \begin{align*}
        \frac{\partial}{\partial A} \left( \frac{1}{n} \| Y - ZA^\top\|_F^2 \right) &= 0 \\
        \frac{1}{n} \frac{\partial}{\partial A} \left( \sum_{i=1}^n \| y_i - z_iA^\top\|^2 \right) &= 0 \\
        \frac{1}{n} \sum_{i=1}^n \frac{\partial}{\partial A} \left( \| y_i - z_iA^\top\|^2 \right) &= 0 \\
        \frac{1}{n} \sum_{i=1}^n \frac{\partial}{\partial A} \left( (y_i - z_iA^\top)^T(y_i - z_iA^\top) \right) &= 0 \\
        \frac{1}{n} \sum_{i=1}^n \frac{\partial}{\partial A} \left( y_i^Ty_i - y_i^Tz_iA^\top - A(z_i^Ty_i) + A(z_i^Tz_i)A^\top \right) &= 0 \\
        \frac{1}{n} \sum_{i=1}^n \frac{\partial}{\partial A} \left( y_i^Ty_i - 2y_i^Tz_iA^\top + A(z_i^Tz_i)A^\top \right) &= 0 \\
        \frac{1}{n} \sum_{i=1}^n \left(-2 y_i^\top z_i + 2z_i^\top z_iA\right) &= 0
    \end{align*}
    Now, we can use the identity constraint $Z^\top Z = I_d$ to get:
    \begin{align*}
        \frac{1}{n} \sum_{i=1}^n \left(-2 y_i^\top z_i + 2z_i^\top z_iA\right) &= 0\\
        \implies \frac{1}{n} \sum_{i=1}^n \left(-2 y_i^\top z_i + 2A\right) &= 0\\
        \implies \frac{1}{n} \sum_{i=1}^n \left(-2 y_i^\top z_i\right) &= -2A\\
        \implies A &= \frac{1}{n} \sum_{i=1}^n y_i^\top z_i
    \end{align*}
    Rearranging and writing this in matrix form, we get $Y = AZ^\top$
    The rank $d$ approximation of $Y$ is given by $Y = U\Sigma V^\top$. \\
    Thus, $Z = U_d$ and $A = \Sigma_dV_d^\top$ where $U_d$ and $V_d$ are the first $d$ columns of $U$ and $V$ respectively will give us the best possible $Z$ and $A$.\\
    In our derivation, we have already assumed that $Z$ has a zero mean and identity covariance and so these constraints are satisfied.\\

    \item First we should subtract $b$, calculated from the training data (just the training data mean) from $x_*$. Let this be $\tilde{x_*}$. Then, we should project $\tilde{x_*}$ onto the subspace spanned by the first $d$ singular vectors of $Z$ to get $z_*$ using $z_* = A^\top \tilde{x_*}$\

\end{enumerate}
\subsection{Experiment (34 points)}

Here we will compare the above three methods on two data sets. 

\begin{itemize}
\item We will implement three variants of PCA:
\begin{enumerate}
    \item "buggy PCA": PCA applied directly on the matrix $X$.
    \item "demeaned PCA": We subtract the mean along each dimension before applying PCA.
    \item "normalized PCA": Before applying PCA, we subtract the mean and scale each dimension so that the sample  mean and standard deviation along each dimension is $0$ and $1$ respectively.
    
\end{enumerate}



\item 
One way to study how well the low dimensional representation $Z$ captures the linear
structure in our data is to project $Z$ back to $D$ dimensions and look at the reconstruction
error. For PCA, if we mapped it to $d$ dimensions via $z = Vx$ then the
reconstruction is $V^\top z$. For the preprocessed versions, we first do this and then
reverse the preprocessing steps as well. For DRO  we just compute $Az + b$.
We will compare all methods by the reconstruction error on the datasets.

\item 
Please implement code for the methods: Buggy PCA (just take the SVD of $X$)
, Demeaned PCA,
Normalized PCA, DRO. In all cases your function should take in
an $n \times d$ data matrix and $d$ as an argument. It should return the
the $d$ dimensional representations, the estimated parameters, and the
reconstructions of these representations in $D$ dimensions. 

\item
You are given two datasets: A two Dimensional dataset with $50$ points 
\texttt{data2D.csv} and a thousand dimensional dataset with $500$ points
\texttt{data1000D.csv}. 

\item
For the $2D$ dataset use $d=1$. For the $1000D$ dataset, you need to choose
$d$. For this, observe the singular values in DRO and see if there is a clear
``knee point" in the spectrum.
Attach any figures/ Statistics you computed to justify your choice.

\item
For the $2D$ dataset you need to attach the a 
plot comparing the orignal points with the reconstructed points for all 4
methods.
For both datasets you should also report the reconstruction errors, that is the squared sum of
differences $\sum_{i=1}^n \|x_i - r(z_i)\|^2$,
where $x_i$'s are the original points and $r(z_i)$ are the $D$ dimensional points
reconstructed from the 
$d$ dimensional representation $z_i$.

\item \textbf{Questions:} After you have completed the experiments, please answer the following questions.
\begin{enumerate}
\item Look at the results for Buggy PCA. The reconstruction error is bad and the
reconstructed points don't seem to well represent the original points. Why is
this? \\
\textbf{Hint: } Which subspace is Buggy PCA trying to project the points
onto?
\item The error criterion we are using is the average squared error 
between the original points and the reconstructed points.
In both examples DRO and demeaned PCA achieves the lowest error among all
methods. 
Is this surprising? Why?
\end{enumerate}

\item Point allocation:
\begin{itemize}
\item Implementation of the three PCA methods: \textbf{(6 Points)}
\item Implementation of DRO: \textbf{(6 points)}
\item Plots showing original points and reconstructed points for 2D dataset for each one of the 4 methods: \textbf{(10 points)}
\item Implementing reconstructions and reporting results for each one of the 4 methods for the 2 datasets: \textbf{(5 points)}
\item Choice of $d$ for $1000D$ dataset and appropriate justification:
\textbf{(3 Points)}
\item Questions \textbf{(4 Points)}
\end{itemize}

\end{itemize}


%\vspace{0.1in}

\vspace{0.2in}

\textbf{Answer format:}  \\
The graph below is what is obtained for the 2D data set when projecting the data onto the first principal component. \\
\begin{figure}[H]
    \centering
    \includegraphics[width=1.\linewidth]{../images/2_3.png}
    \caption{Reconstructed points for 2D data set}
    \label{fig:rec_2d}
\end{figure}
The reconstruction loss for the 2D dataset is given as
\begin{verbatim}
    Buggy reconstruction error dataset1: 44.34515418673971
    Demeaned reconstruction error dataset1: 0.5003042814256459
    Normalized reconstruction error dataset1: 2.473604172738536
    DRO reconstruction error dataset1: 0.5003042814256461
\end{verbatim}


To choose the knee point, I plot the singular values obtained vs the index and pick \textbf{index = 30} as the knee point. \\
\begin{figure}[H]
    \centering
    \includegraphics[width=0.5\linewidth]{../images/2_3_singular_values.png}
    \caption{Knee point by plotting the singular values}
    \label{fig:rec_1000d}
\end{figure}

Then, using this as the knee point , I get the following reconstruction losses
\begin{verbatim}
    Buggy reconstruction error dataset1: 136384.9746109073
    Demeaned reconstruction error dataset1: 135697.80211063704
    Normalized reconstruction error dataset1: 136029.46338770172
    DRO reconstruction error dataset1: 135697.80211063707    
\end{verbatim}


\textbf{Questions: }\\
\begin{enumerate}
    \item In buggy PCA, we use the SVD of the data matrix $X$ to get the principal components. However, since the data is not centered, the principal components are not the directions of maximum variance since $XX^T$ is proportional to sample covariance only when the data is centered. Thus, the reconstruction error is high, and the reconstructed points do not represent the original points well.
    \item This is not surprising since the data is zero meaned we are projecting onto the subspace of maximum variance. Thus, the reconstruction error is low.
\end{enumerate}




\bibliographystyle{apalike}
\end{document}


