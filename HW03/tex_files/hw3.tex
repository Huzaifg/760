\documentclass[a4paper]{article}
\usepackage{geometry}
\usepackage{graphicx}
\usepackage{natbib}
\usepackage{amsmath}
\usepackage{amssymb}
\usepackage{amsthm}
\usepackage{paralist}
\usepackage{epstopdf}
\usepackage{tabularx}
\usepackage{longtable}
\usepackage{multirow}
\usepackage{multicol}
\usepackage[hidelinks]{hyperref}
\usepackage{fancyvrb}
\usepackage{algorithm}
\usepackage{algorithmic}
\usepackage{float}
\usepackage{paralist}
\usepackage[svgname]{xcolor}
\usepackage{enumerate}
\usepackage{array}
\usepackage{times}
\usepackage{url}
\usepackage{fancyhdr}
\usepackage{comment}
\usepackage{environ}
\usepackage{times}
\usepackage{textcomp}
\usepackage{caption}
\usepackage{multirow}


\urlstyle{rm}

\setlength\parindent{0pt} % Removes all indentation from paragraphs
\theoremstyle{definition}
\newtheorem{definition}{Definition}[]
\newtheorem{conjecture}{Conjecture}[]
\newtheorem{example}{Example}[]
\newtheorem{theorem}{Theorem}[]
\newtheorem{lemma}{Lemma}
\newtheorem{proposition}{Proposition}
\newtheorem{corollary}{Corollary}


\floatname{algorithm}{Procedure}
\renewcommand{\algorithmicrequire}{\textbf{Input:}}
\renewcommand{\algorithmicensure}{\textbf{Output:}}
\newcommand{\abs}[1]{\lvert#1\rvert}
\newcommand{\norm}[1]{\lVert#1\rVert}
\newcommand{\RR}{\mathbb{R}}
\newcommand{\CC}{\mathbb{C}}
\newcommand{\Nat}{\mathbb{N}}
\newcommand{\br}[1]{\{#1\}}
\DeclareMathOperator*{\argmin}{arg\,min}
\DeclareMathOperator*{\argmax}{arg\,max}
\renewcommand{\qedsymbol}{$\blacksquare$}

\definecolor{dkgreen}{rgb}{0,0.6,0}
\definecolor{gray}{rgb}{0.5,0.5,0.5}
\definecolor{mauve}{rgb}{0.58,0,0.82}

\newcommand{\Var}{\mathrm{Var}}
\newcommand{\Cov}{\mathrm{Cov}}

\newcommand{\vc}[1]{\boldsymbol{#1}}
\newcommand{\xv}{\vc{x}}
\newcommand{\Sigmav}{\vc{\Sigma}}
\newcommand{\alphav}{\vc{\alpha}}
\newcommand{\muv}{\vc{\mu}}
\newcommand{\rr}{\mathbb{R}}
\newcommand{\red}[1]{\textcolor{red}{#1}}



\newcommand{\diag}{\operatorname{diag}}
\newcommand{\innp}[1]{\left\langle #1 \right\rangle}
\newcommand{\bdot}[1]{\mathbf{\dot{ #1 }}}
\newcommand{\OPT}{\operatorname{OPT}}
\newcommand{\mA}{\mathbf{A}}
\newcommand{\mP}{\mathbf{P}}
\newcommand{\mLambda}{\mathbf{\Lambda}}
\newcommand{\ones}{\mathds{1}}
\newcommand{\zeros}{\textbf{0}}
\newcommand{\vx}{\mathbf{x}}
\newcommand{\vtheta}{\mathbf{\theta}}
\newcommand{\vp}{\mathbf{p}}
\newcommand{\dd}{\mathrm{d}}
\newcommand{\cx}{\mathcal{X}}
\newcommand{\cy}{\mathcal{Y}}
\newcommand{\cc}{\mathcal{C}}
\newcommand{\cz}{\mathcal{Z}}
\newcommand{\vxh}{\mathbf{\hat{x}}}
\newcommand{\vyh}{\mathbf{\hat{y}}}
\newcommand{\vzh}{\mathbf{\hat{z}}}
\newcommand{\vy}{\mathbf{y}}
\newcommand{\vz}{\mathbf{z}}
\newcommand{\vv}{\mathbf{v}}
\newcommand{\ve}{\mathbf{e}}
\newcommand{\va}{\mathbf{a}}
\newcommand{\vw}{\mathbf{w}}
\newcommand{\vvh}{\mathbf{\hat{v}}}
\newcommand{\vb}{\mathbf{b}}
\newcommand{\vg}{\mathbf{g}}
\newcommand{\vu}{\mathbf{u}}
\newcommand{\vub}{\overline{\mathbf{u}}}
\newcommand{\vuh}{\hat{\mathbf{u}}}
\newcommand{\veta}{\bm{\eta}}
\newcommand{\vetah}{\bm{\hat{\eta}}}
\newcommand{\defeq}{\stackrel{\mathrm{\scriptscriptstyle def}}{=}}
\newcommand{\etal}{\textit{et al}.}
\newcommand{\tnabla}{\widetilde{\nabla}}
\newcommand{\tE}{\widetilde{E}}
\newcommand{\bmat}[1]{\begin{bmatrix}#1\end{bmatrix}} 
\newcommand{\inner}[2]{\langle#1,#2\rangle}


\def\x{\mathbf x}
\def\y{\mathbf y}
\def\w{\mathbf w}
\def\v{\mathbf v}
\def\E{\mathbb E}
\def\V{\mathbb V}

% TO SHOW SOLUTIONS, include following (else comment out):
\newenvironment{soln}{
    \leavevmode\color{blue}\ignorespaces
}{}


\hypersetup{
%    colorlinks,
    linkcolor={red!50!black},
    citecolor={blue!50!black},
    urlcolor={blue!80!black}
}

\geometry{
  top=1in,            % <-- you want to adjust this
  inner=1in,
  outer=1in,
  bottom=1in,
  headheight=3em,       % <-- and this
  headsep=2em,          % <-- and this
  footskip=3em,
}


\pagestyle{fancyplain}
\lhead{\fancyplain{}{Homework 3}}
\rhead{\fancyplain{}{CS 760 Machine Learning}}
\cfoot{\thepage}

\title{\textsc{Homework 3}} % Title

%%% NOTE:  Replace 'NAME HERE' etc., and delete any "\red{}" wrappers (so it won't show up as red)

\author{
\red{$>>$Huzaifa Mustafa Unjhawala$<<$} \\
\red{$>>$ID HERE$<<$}\\
} 

\date{}

\begin{document}

\maketitle 


\textbf{Instructions:} 
Use this latex file as a template to develop your homework. Submit your homework on time as a single pdf file to Canvas. Late submissions may not be accepted. Please wrap your code and upload to a public GitHub repo, then attach the link below the instructions so that we can access it. You can choose any programming language (i.e. python, R, or MATLAB). Please check Piazza for updates about the homework.

\section{Questions (50 pts)}
\begin{enumerate}
\item (9 pts) Explain whether each scenario is a classification or regression problem. And, provide the number of data points ($n$) and the number of features ($p$).

\begin{enumerate}
	\item (3 pts) We collect a set of data on the top 500 firms in the US. For each firm we record profit, number of employees, industry and the CEO salary. We are interested in predicting CEO salary with given factors.
	
	\begin{soln}  This is a regression problem as the salary of a CEO $\in \rr$ \end{soln}
	
	\item (3 pts) We are considering launching a new product and wish to know whether it will be a success or a failure. We collect data on 20 similar products that were previously launched. For each product we have recorded whether it was a success or failure, price charged for the product, marketing budget, competition price, and ten other variables.
	
	\begin{soln}  Based on the feature set we need to predict whether the product will be either a success or a failure. Thus, this is a classification problem.\end{soln}
	
	\item (3 pts) We are interesting in predicting the \% change in the US dollar in relation to the weekly changes in the world stock markets. Hence we collect weekly data for all of 2012. For each week we record the \% change in the dollar, the \% change in the US market, the \% change in the British market, and the \% change in the German market.
	
	\begin{soln} This is a regression problem as we would like to predict the percentage change which $\in \rr$ \end{soln}
	
\end{enumerate}

\item (6 pts) The table below provides a training data set containing six observations, three predictors, and one qualitative response variable.

\begin{center}
	\begin{tabular}{ c  c  c  c}
		\hline
		$X_{1}$ & $X_{2}$ & $X_{3}$ & $Y$ \\ \hline
		0 & 3 & 0 & Red \\
		2 & 0 & 0 & Red \\
		0 & 1 & 3 & Red \\
		0 & 1 & 2 & Green \\
		-1 & 0 & 1 & Green \\
		1 & 1 & 1 & Red  \\
		\hline
	\end{tabular}
\end{center}

Suppose we wish to use this data set to make a prediction for $Y$ when $X_{1} = X_{2} = X_{3} = 0$ using K-nearest neighbors.

\begin{enumerate}
	\item (2 pts) Compute the Euclidean distance between each observation and the test point, $X_{1} = X_{2} = X_{3}=0$.
 
	\begin{soln}  
		\begin{center}
			\begin{tabular}{ c  c}
				\hline
				S.No. & Distance \\ \hline
				1 & 3 \\
				2 & 2 \\
				3 & 3.162 \\
				4 & 2.236 \\
				5 & 1.414 \\
				6 & 1.732 \\
				\hline
			\end{tabular}
		\end{center}
	\end{soln}
 
	\item (2 pts) What is our prediction with $K=1$? Why?
	
	\begin{soln}
	5th point has the least distance from the test point. Hence, the prediction is Green.
	\end{soln}
	
	\item (2 pts) What is our prediction with $K=3$? Why?
	
	\begin{soln}
	The three closest points are 5th, 6th and 2nd. That is 2 reds and 1 green. Thus, by majority vote, the prediction is red.
	\end{soln}

\end{enumerate}

\item (12 pts) When the number of features $p$ is large, there tends to be a deterioration in the performance of KNN and other local approaches that perform prediction using only observations that are near the test observation for which a prediction must be made. This phenomenon is known as the curse of dimensionality, and it ties into the fact that non-parametric approaches often perform poorly when $p$ is large.

\begin{enumerate}
	\item (2pts) Suppose that we have a set of observations, each with measurements on $p=1$ feature, $X$. We assume that $X$ is uniformly (evenly) distributed on [0, 1]. Associated with each observation is a response value. Suppose that we wish to predict a test observation’s response using only observations that are within 10\% of the range of $X$ closest to that test observation. For instance, in order to predict the response for a test observation with $X=0.6$, we will use observations in the range [0.55, 0.65]. On average, what fraction of the available observations will we use to make the prediction?
	
	\begin{soln} 
		10\% since its uniformly distributed.
	\end{soln}
	
	
	\item (2pts) Now suppose that we have a set of observations, each with measurements on $p =2$ features, $X1$ and $X2$. We assume that predict a test observation’s response using only observations that $(X1,X2)$ are uniformly distributed on [0, 1] × [0, 1]. We wish to are within 10\% of the range of $X1$ and within 10\% of the range of $X2$ closest to that test observation. For instance, in order to predict the response for a test observation with $X1 =0.6$ and $X2 =0.35$, we will use observations in the range [0.55, 0.65] for $X1$ and in the range [0.3, 0.4] for $X2$. On average, what fraction of the available observations will we use to make the prediction?
	
	\begin{soln}
		10\% of the range of $X1$ and 10\% of the range of $X2$. Thus, 1\% of the observations.
	\end{soln}
	
	\item (2pts) Now suppose that we have a set of observations on $p = 100$ features. Again the observations are uniformly distributed on each feature, and again each feature ranges in value from 0 to 1. We wish to predict a test observation’s response using observations within the 10\% of each feature’s range that is closest to that test observation. What fraction of the available observations will we use to make the prediction?
	
	\begin{soln}
		That will be $0.1^{100} * 100 \%$
	\end{soln}
	
	\item (3pts) Using your answers to parts (a)–(c), argue that a drawback of KNN when p is large is that there are very few training observations “near” any given test observation.
	
	\begin{soln}
	Since the number of observations close to us keeps decreasing exponentially as the dimension of the feature space keeps increasing, the predicted label will be based on very few observations. This will lead to a high variance in the prediction in this high dimensional feature space.
	\end{soln}
	
	\item (3pts) Now suppose that we wish to make a prediction for a test observation by creating a $p$-dimensional hypercube centered around the test observation that contains, on average, 10\% of the training observations. For $p =$1, 2, and 100, what is the length of each side of the hypercube? Comment what happens to the length of the sides as $\lim_{{p \to \infty}}$.
	
	\begin{soln}
		As $p \to \infty$, the length of the sides of the hypercube $\to 0$.
	\end{soln}
	
\end{enumerate}

\item (6 pts) Supoose you trained a classifier for a spam detection system. The prediction result on the test set is summarized in the following table.
\begin{center}
	\begin{tabular}{l l | l l}
		&          & \multicolumn{2}{l}{Predicted class} \\
		&          & Spam           & not Spam           \\
		\hline
		\multirow{2}{*}{Actual class} & Spam     & 8              & 2                  \\
		& not Spam & 16             & 974               
	\end{tabular}
\end{center}

Calculate
\begin{enumerate}
	\item
	\begin{soln} 
	Accuracy is a number of correct predictions by the total data points $ = \frac{974 + 8}{1000} = 0.982$
	\end{soln}
	\item 
	\begin{soln}
	Precision is $\frac{TP}{TP + FP}$ where TP is True Positive and FP is False Positive. Thus, $\frac{8}{8 + 16} = 0.333$
	\end{soln}
	\item
	\begin{soln}
	Recall is $\frac{TP}{TP + FN}$ where TP is True Positive and FN is False Negative. Thus, $\frac{8}{8 + 2} = 0.8$
	\end{soln}
\end{enumerate}


\item (9pts) Again, suppose you trained a classifier for a spam filter. The prediction result on the test set is summarized in the following table. Here, "+" represents spam, and "-" means not spam.

\begin{center}
\begin{tabular}{ c  c }
\hline
Confidence positive & Correct class \\ \hline
0.95 & + \\
0.85 & + \\
0.8 & - \\
0.7 & + \\
0.55 & + \\
0.45 & - \\
0.4 & + \\
0.3 & + \\
0.2 & - \\
0.1 & - \\
\hline
\end{tabular}
\end{center}

\begin{enumerate}
	\item (6pts) Draw a ROC curve based on the above table.
	
	\begin{soln}  
	\begin{figure}[H]
		\centering
		\includegraphics[width=0.5\linewidth]{../images/1_5.png}
		\caption{ROC Curve}
		\label{fig:roc}
	\end{figure}
	
	\end{soln}
	
	\item (3pts) (Real-world open question) Suppose you want to choose a threshold parameter so that mails with confidence positives above the threshold can be classified as spam. Which value will you choose? Justify your answer based on the ROC curve.
	
	\begin{soln}
	I will choose the threshold to be $0.85$ as this will ensure 0 false positives (from ROC curve). This will ensure that the spam detector will only put a mail to spam if sure that it is spam.
	\end{soln}
\end{enumerate}

\item (8 pts) In this problem, we will walk through a single step of the gradient descent algorithm for logistic regression. As a reminder,
$$\hat{y} = f(x, \theta)$$
$$f(x;\theta) = \sigma(\theta^\top x)$$
$$\text{Cross entropy loss } L(\hat{y}, y) = -[y \log  \hat{y} + (1-y)\log(1-\hat{y})]$$
$$\text{The single update step } \theta^{t+1} = \theta^{t} - \eta \nabla_{\theta} L(f(x;\theta), y) $$



\begin{enumerate}
	\item (4 pts) Compute the first gradient $\nabla_{\theta} L(f(x;\theta), y)$.
	
	\begin{soln}
		\begin{align*}
		L(f(x;\theta), y) &= -\Big[y \log(\hat{y}) + (1-y) \log(1- \hat{y})\Big] \\
		&= -\Big[y \log(\sigma(\theta^\top x)) + (1-y) \log(1- \sigma(\theta^\top x))\Big] \\
		\end{align*}
		Also, for a sigmoid function, we know its derivative is $\sigma(x)(1-\sigma(x))$. Thus, using chain rule:
		\begin{align*}
			\nabla_{\vtheta} \sigma(\vtheta^\top \vx) &= \sigma(\vtheta^\top \vx)(1-\sigma(\vtheta^\top \vx))\vx \\
		\end{align*}
		Using this, we can take the derivative of the loss function with chain rule as follows:
		\begin{align*}
		\nabla_{\vtheta} L(f(x;\vtheta), y) &= -\Big[y \frac{\sigma(\vtheta^\top \vx)(1-\sigma(\vtheta^\top \vx))}{\sigma(\vtheta^\top \vx)}\vx + (1-y) \frac{\sigma(\vtheta^\top \vx)(1-\sigma(\vtheta^\top \vx))}{1-\sigma(\vtheta^\top \vx)}\vx\Big] \\
		&= -\Big[y (1-\sigma(\vtheta^\top \vx))\vx - (1-y) \sigma(\vtheta^\top \vx)\vx\Big] \\
		\end{align*}
		Cleaning this up a bit and taking common denominator:
		\begin{align*}
		\nabla_{\vtheta} L(f(x;\vtheta), y) &= -\Big[\frac {y (1-\sigma(\vtheta^\top \vx)) - (1-y) \sigma(\vtheta^\top \vx)}{\sigma(\vtheta^\top \vx)(1-\sigma(\vtheta^\top \vx))}\Big] \sigma(\vtheta^\top \vx)(1-\sigma(\vtheta^\top \vx))\vx \\
		\end{align*}
		denominator cancels out:
		\begin{align*}
		\nabla_{\vtheta} L(f(x;\vtheta), y) &= -\Big[y (1-\sigma(\vtheta^\top \vx)) - (1-y) \sigma(\vtheta^\top \vx)\Big] \vx \\
		\end{align*}
		Expanding we get:
		\begin{align*}
		\nabla_{\vtheta} L(f(x;\vtheta), y) &= -\Big[y - y\sigma(\vtheta^\top \vx) - \sigma(\vtheta^\top \vx) + y\sigma(\vtheta^\top \vx)\Big] \vx \\
		&= -\Big[y - \sigma(\vtheta^\top \vx)\Big] \vx \\
		\end{align*}
	\end{soln}
	
	\item (4 pts)
 Now assume a two dimensional input. After including a bias parameter for the first dimension, we will have $\theta\in\mathbb{R}^3$.
$$ \text{Initial parameters : }  \theta^{0}=[0, 0, 0]$$
$$ \text{Learning rate }\eta=0.1$$
$$ \text{data example : } x=[1, 3, 2], y=1$$
Compute the updated parameter vector $\theta^{1}$ from the single update step.
	
	\begin{soln}
		$\vtheta^\top \vx = 0$. And $\sigma(0) = 0.5$ \\
		The gradient then evaluates to
		\begin{align*}
		\nabla_{\vtheta} L(f(x;\vtheta), y) &= -\Big[y - \sigma(\vtheta^\top \vx)\Big] \vx \\
		&= -\Big[1 - \sigma(\vtheta^\top \vx)\Big] \vx \\
		&= -\Big[1 - 0.5\Big] \vx \\
		&= -0.5 \vx \\
		&= [-0.5, -1.5, -1] \\
		\end{align*}

		Therefore, the updated parameter vector $\vtheta^1$ is:
		\begin{align*}
		\vtheta^1 &= \vtheta^0 - \eta \nabla_{\vtheta} L(f(x;\vtheta), y) \\
		&= [0, 0, 0] - 0.1 [-0.5, -1.5, -1] \\
		&= [0.05, 0.15, 0.1] \\
		\end{align*}
	\end{soln}
\end{enumerate}
\end{enumerate}

\section{Programming (50 pts)}
\begin{enumerate}
	\item (10 pts) Use the whole D2z.txt as training set.  Use Euclidean distance (i.e. $A=I$).
	Visualize the predictions of 1NN on a 2D grid $[-2:0.1:2]^2$.
	That is, you should produce test points whose first feature goes over $-2, -1.9, -1.8, \ldots, 1.9, 2$, so does the second feature independent of the first feature.
	You should overlay the training set in the plot, just make sure we can tell which points are training, which are grid.
	
	\begin{soln}  
		\begin{figure}[H]
			\centering
			\includegraphics[width=0.5\linewidth]{../images/D2z.png}
			\caption{1NN on D2z.txt}
			\label{fig:1nn}
		\end{figure}
	\end{soln}	

	\begin{itemize}
		\item Task: spam detection
		\item The number of rows: 5000
		\item The number of features: 3000 (Word frequency in each email)
		\item The label (y) column name: `Predictor'
		\item For a single training/test set split, use Email 1-4000 as the training set, Email 4001-5000 as the test set.
		\item For 5-fold cross validation, split dataset in the following way.
		\begin{itemize}
			\item Fold 1, test set: Email 1-1000, training set: the rest (Email 1001-5000)
			\item Fold 2, test set: Email 1000-2000, training set: the rest
			\item Fold 3, test set: Email 2000-3000, training set: the rest
			\item Fold 4, test set: Email 3000-4000, training set: the rest
			\item Fold 5, test set: Email 4000-5000, training set: the rest			
		\end{itemize}
	\end{itemize}
	
	\item (8 pts) Implement 1NN, Run 5-fold cross validation. Report accuracy, precision, and recall in each fold.
	
	\begin{soln}
		\begin{verbatim}
			Accuracy:  [0.825, 0.855, 0.863, 0.854, 0.775]
			Precision:  [0.6536, 0.6896, 0.7220, 0.7215, 0.6051]
			Recall:  [0.8210, 0.8664, 0.8415, 0.8197, 0.7614]
			
		\end{verbatim}
	\end{soln}
	
	\item (12 pts) Implement logistic regression (from scratch). Use gradient descent (refer to question 6 from part 1) to find the optimal parameters. You may need to tune your learning rate to find a good optimum. Run 5-fold cross validation. Report accuracy, precision, and recall in each fold.
	
	\begin{soln}
	Using learning rate of 0.01 for 1000 epochs
	\begin{verbatim}
		Accuracy:  [0.908, 0.882, 0.875, 0.876, 0.848]
		Precision:  [0.8409, 0.8697, 0.8955, 0.8269, 0.7810]
		Recall:  [0.8350, 0.6750, 0.6338, 0.7312, 0.6993]
		

	\end{verbatim}
	\end{soln}
	
	\item (10 pts) Run 5-fold cross validation with kNN varying k (k=1, 3, 5, 7, 10). Plot the average accuracy versus k, and list the average accuracy of each case. \\
	Expected figure looks like this.

	\begin{soln} 
		\begin{figure}[H]
			\centering
			\includegraphics[width=0.5\linewidth]{../images/2_4.png}
			\caption{Accuracy vs k}
			\label{fig:knn_4}
		\end{figure}
	\end{soln}
	
	\item (10 pts) Use a single training/test setting. Train kNN (k=5) and logistic regression on the training set, and draw ROC curves based on the test set. \\
	Expected figure looks like this.
	Note that the logistic regression results may differ.
	
	\begin{soln}  Solution goes here. \end{soln}
	
\end{enumerate}
\bibliographystyle{apalike}
\end{document}
